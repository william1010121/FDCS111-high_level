\documentclass[12pt,oneside]{article}
\usepackage{xeCJK}
\xeCJKsetup{AutoFakeBold=true, AutoFakeSlant=true}
\setCJKmainfont{新細明體}
\setmainfont{Times New Roman}
%\setmainfont{標楷體}
%\usepackage{ctex}

\usepackage{bbding} %打勾叉
\usepackage{xcolor}
\usepackage{amsmath} %數學庫
\usepackage{amssymb}
\usepackage{setspace}
\usepackage{fancyhdr} %排版用,頁數
\usepackage[colorlinks=true, linkcolor=black, citecolor=black, urlcolor=blue]{hyperref}

\usepackage{graphicx} %插入图片的宏包
\usepackage{float} %设置图片浮动位置的宏包
\usepackage{subfigure} %插入多图时用子图显示的宏包
\usepackage{chngpage}
\usepackage{emptypage}


\usepackage[margin=2cm]{geometry} %上下左右的流空
\usepackage{diagbox} %表格的斜線標題

\usepackage{spacingtricks} %表格間距
\usepackage{indentfirst}
\setlength{\parindent}{2em}

%------ c++顏色 (我不會 :poop:)) ---- %
\usepackage{listings}
\definecolor{dkgreen}{rgb}{0,0.6,0}
\definecolor{gray}{rgb}{0.5,0.5,0.5}
\definecolor{mauve}{rgb}{0.58,0,0.82}
\lstset{
	numbers=left,  
	frame=tb,
	aboveskip=3mm,
	belowskip=3mm,
	showstringspaces=false,
	columns=flexible,
	framerule=1pt,
	rulecolor=\color{gray!35},
	backgroundcolor=\color{gray!5},
	basicstyle={\ttfamily},
	numberstyle=\tiny\color{gray},
	keywordstyle=\color{blue},
	commentstyle=\color{dkgreen},
	stringstyle=\color{mauve},
	breaklines=true,
	breakatwhitespace=true,
	tabsize=3,
}
%原文链接:https://blog.csdn.net/weixin_50012998/article/details/109455978
%------ c++顏色 (我不會 :poop:)) ---- %
%--------------------color---------------------------%
\definecolor{group_color_1}{HTML}{1D4765}
\definecolor{group_color_2}{HTML}{E5e6e1}
\definecolor{group_color_3}{HTML}{A7d1d9}
\definecolor{group_color_4}{HTML}{bebbb7}

\definecolor{TreeEdge}{HTML}{1B998B}
\definecolor{BackEdge}{HTML}{4E6EE4}
\definecolor{CrossEdge}{HTML}{8544DA}
\definecolor{ForwardEdge}{HTML}{534d56}

%--------------------color---------------------------%


%畫流程圖
\usepackage{tikz, mathpazo}
\usetikzlibrary{shapes.geometric, arrows}
\usetikzlibrary{shapes, shapes.multipart}
\usetikzlibrary{calc, positioning}
\usetikzlibrary{backgrounds}
\usetikzlibrary{fit}


\renewcommand{\baselinestretch}{1.5}
\renewcommand{\contentsname}{目錄}

\title{ 競程觀念 }
\author{ 進階教學組 }
\date{ \today }


%----------------初始化------------------%


%---------縮放------------%
\tikzset{global scale/.style={
    scale=#1,
    every node/.append style={scale=#1}
  }
}
\tikzset{Decision_shape/.style={
    trapezium, draw, trapezium left angle=60,
    trapezium right angle=-60
}
}

\tikzset{
    basic box/.style = {
        shape=rectangle,
        align=center, 
        draw=#1,
        dashed,
        thick, 
        rounded corners
    }
}

%---------縮放------------%



%--------流程圖樣式---------------%
\tikzstyle{startstop} = [rectangle, rounded corners, minimum width=3cm, minimum height=1cm,text centered,text=white, draw=black, fill=group_color_1, text width=3cm]
\tikzstyle{io} = [trapezium, trapezium left angle=70, trapezium right angle=110, minimum width=3cm, minimum height=1cm, text centered, draw=black, fill=group_color_2, text width=3cm]
\tikzstyle{process} = [rectangle, minimum width=3cm, minimum height=1cm, text centered, draw=black, fill=group_color_3, text width=3cm]
\tikzstyle{decision} = [Decision_shape, minimum width=3cm, minimum height=1cm, text centered, draw=black, fill=group_color_4, text width=3cm]
%--------流程圖樣式---------------%

%------ 圖論樣式 ------ %
\tikzstyle{node} = [circle, minimum width=0cm, minimum height=0cm, text centered, draw]
\tikzstyle{arrow} = [->, thick]
%------ 圖論樣式 ------ %


%---------設置最大節點䥗度------------%
\tikzset {
MaxWidth3/.style={
    text width=3cm
}
}
%---------設置最大節點䥗度------------%

\begin{document}
\maketitle
\thispagestyle{empty}
\clearpage

\tableofcontents
\setcounter{page}{0}

\clearpage
\pagestyle{fancy}
\rhead{程式考試、競賽觀念}

\section{賽制介紹}

\begin{center}
\begin{tabular}{|l|*{4}{|c}|} 
    \hline
    \diagbox{賽制}{特性} & 賽中評測 & 排名 & 懲罰(penaty) & 比賽、考試 \\\hline
    ACM & \Checkmark & \Checkmark & \Checkmark  & \href{https://codeforces.com/}{codeforces}\\\hline
    OI & \XSolidBrush & \XSolidBrush & \XSolidBrush & \href{https://apcs.csie.ntnu.edu.tw/}{APCS} \\\hline
    IOI & \Checkmark & $\Delta$ & \XSolidBrush  & TOI \\\hline
\end{tabular}
\end{center}


% 賽中評測 是否可以看到排名 懲罰(penaty)
% ACM 
% OI
% IOI
%


 

\section{需掌握的知識}

\subsection{APCS}
%\href{https://apcs.csie.ntnu.edu.tw/index.php/questionstypes/}{考試範圍}
\begin{enumerate}
    \item 初階班的所有東西
    \item 函數呼叫與遞迴
    \item 基礎資料結構
        \begin{itemize}
            \item queues
            \item stacks
            \item tree
            \item graph
        \end{itemize}
    \item 基礎演算法
    \begin{itemize}
        \item sorting
        \item searching
        \item greedy
        \item dynamic programming
    \end{itemize}
\end{enumerate}
%,字串 (string)
%函數呼叫與遞迴 (function call and recursion)
%基礎資料結構 (basic data structures),包括:佇列 (queues),堆疊 (stacks),樹狀圖 (tree),圖形 (graph)
%基礎演算法 (basic algorithms),包括:排序 (sorting),搜尋 (searching),貪心法則 (greedy method),
%動態規劃 (dynamic programming)

\subsection{IOI/TOI}
這個就什麼都不嫌多,
因為可能有些人的目標不是這個,
所以就有興趣的再來找我,
我可以請學長教我們。

\section{團體必賽的工作分配}
如果是要比\href{https://contest.cc.ntu.edu.tw/npsc2021/}{NPSC}或著是\href{https://www.tw-ytp.org/}{YTP}這類需要組隊的比賽時,
分工就會變得很重要,
借用\href{https://hackmd.io/HAShR4J-Q7yhRWboqvLPJA?view}{前一屆的整理},
通常會需要有\\
\begin{enumerate}
    \item 一個打code的人
    \item 一個算數學的人
    \item 一個通靈的人
\end{enumerate}

\subsection{打\textbf{code}}
就需要把抽象的解題方式化為具體的程式碼,
通常打字速度要足夠快,
大概$6, 70$左右應該就足以應付了。
在打code的時候要盡量的讓其他人能夠看懂,
所以對於一些地方可能需要著墨一下,
像是\textbf{函式化編程}抑或是\textbf{變數名稱}還有\textbf{宏樣式},
都需要先跟隊友討論過,
確保在比賽的時候不會出現問題。

\subsection{算數學}
對於每一個比賽,
通常都會有數論題,
算數學的就是需要把數論題解決,
無論是觀察性質亦或是簡化計算,
都可以由算數學的來完成,
有時候也可以透過一些數學的公式抑或是證明來讓複雜度減少。

\subsection{通靈}
比賽時,
常常會出現一些優美的做法,
可能可以一下子讓複雜度下降一個量級,
從$NA0\%$直接變$AC100\%$,
這就是通靈的工作,
他需要想出題目的解法,
然後把他表示給數學家,
確認正確性以後再交給coder打出來。
可能需要比較天馬行空一些,
抑或著練過了許多題目,
可以快速的抓出題目的特性。

\section{如何debug}
\subsection{黃色鴨鴨法}
如同字面意思,
你可以想像自己面前有一個黃色鴨鴨,
就對鴨鴨講述你的解法和每個code區間的作用,
不用擔心講的很爛,
黃色鴨鴨很有包容心的。

使用黃色鴨鴨法可以讓你重新整理自己的思緒,
跟費曼學習法有一曲同工之妙,
只要熟練掌握,
相信無論在哪裡都會很實用的。

\subsection{追蹤變數}
你可以把程式的變數輸出出來看是否符合正確的過程,
如果發現錯誤你就可以開始\textbf{二分搜}錯誤的地方,
這通常花費的時間比較多一些,

為了節省時間,
你可以把輸出變成一個函式。
不過這需要用到\textbf{template},
所以就先放到以後了。


\section{引用文章}
\href{https://zhuanlan.zhihu.com/p/129311302}{编程比赛三大赛制介绍(ACM赛制、OI赛制、IOI赛制)}
\href{https://web.ntnu.edu.tw/~algo/Activity.html}{演算法筆記}


\end{document}


