\documentclass[12pt,oneside]{article}
\usepackage{xeCJK}
\xeCJKsetup{AutoFakeBold=true, AutoFakeSlant=true}
\setCJKmainfont{新細明體}
\setmainfont{Times New Roman}
%\setmainfont{標楷體}
%\usepackage{ctex}

\usepackage{xcolor}
\usepackage{amsmath} %數學庫
\usepackage{setspace}
\usepackage{fancyhdr} %排版用,頁數
\usepackage[colorlinks=true, linkcolor=black, citecolor=black, urlcolor=blue]{hyperref}
\usepackage{indentfirst}
\setlength{\parindent}{2em}

\usepackage{graphicx} %插入图片的宏包
\usepackage{float} %设置图片浮动位置的宏包
\usepackage{subfigure} %插入多图时用子图显示的宏包
\usepackage{chngpage}
\usepackage{emptypage}


\usepackage[margin=2cm]{geometry} %上下左右的流空
\usepackage{diagbox} %表格的斜線標題

\usepackage{spacingtricks} %表格間距

%------ c++顏色 (我不會 :poop:)) ---- %
\usepackage{listings}
\definecolor{dkgreen}{rgb}{0,0.6,0}
\definecolor{gray}{rgb}{0.5,0.5,0.5}
\definecolor{mauve}{rgb}{0.58,0,0.82}
\lstset{
	numbers=left,  
	frame=tb,
	aboveskip=3mm,
	belowskip=3mm,
	showstringspaces=false,
	columns=flexible,
	framerule=1pt,
	rulecolor=\color{gray!35},
	backgroundcolor=\color{gray!5},
	basicstyle={\ttfamily},
	numberstyle=\tiny\color{gray},
	keywordstyle=\color{blue},
	commentstyle=\color{dkgreen},
	stringstyle=\color{mauve},
	breaklines=true,
	breakatwhitespace=true,
	tabsize=3,
}
%原文链接:https://blog.csdn.net/weixin_50012998/article/details/109455978
%------ c++顏色 (我不會 :poop:)) ---- %
%--------------------color---------------------------%
\definecolor{group_color_1}{HTML}{1D4765}
\definecolor{group_color_2}{HTML}{E5e6e1}
\definecolor{group_color_3}{HTML}{A7d1d9}
\definecolor{group_color_4}{HTML}{bebbb7}

\definecolor{TreeEdge}{HTML}{1B998B}
\definecolor{BackEdge}{HTML}{4E6EE4}
\definecolor{CrossEdge}{HTML}{8544DA}
\definecolor{ForwardEdge}{HTML}{534d56}

%--------------------color---------------------------%


%畫流程圖
\usepackage{tikz, mathpazo}
\usetikzlibrary{shapes.geometric, arrows}
\usetikzlibrary{shapes, shapes.multipart}
\usetikzlibrary{calc, positioning}
\usetikzlibrary{backgrounds}
\usetikzlibrary{fit}


\renewcommand{\baselinestretch}{1.5}
\renewcommand{\contentsname}{目錄}

\title{ Tarjan }
\author{ 進階教學組 }
\date{ \today }


%----------------初始化------------------%


%---------縮放------------%
\tikzset{global scale/.style={
    scale=#1,
    every node/.append style={scale=#1}
  }
}
\tikzset{Decision_shape/.style={
    trapezium, draw, trapezium left angle=60,
    trapezium right angle=-60
}
}

\tikzset{
    basic box/.style = {
        shape=rectangle,
        align=center, 
        draw=#1,
        dashed,
        thick, 
        rounded corners
    }
}

%---------縮放------------%



%--------流程圖樣式---------------%
\tikzstyle{startstop} = [rectangle, rounded corners, minimum width=3cm, minimum height=1cm,text centered,text=white, draw=black, fill=group_color_1, text width=3cm]
\tikzstyle{io} = [trapezium, trapezium left angle=70, trapezium right angle=110, minimum width=3cm, minimum height=1cm, text centered, draw=black, fill=group_color_2, text width=3cm]
\tikzstyle{process} = [rectangle, minimum width=3cm, minimum height=1cm, text centered, draw=black, fill=group_color_3, text width=3cm]
\tikzstyle{decision} = [Decision_shape, minimum width=3cm, minimum height=1cm, text centered, draw=black, fill=group_color_4, text width=3cm]
%--------流程圖樣式---------------%

%------ 圖論樣式 ------ %
\tikzstyle{node} = [circle, minimum width=0cm, minimum height=0cm, text centered, draw]
\tikzstyle{arrow} = [->, thick]
%------ 圖論樣式 ------ %


%---------設置最大節點䥗度------------%
\tikzset {
MaxWidth3/.style={
    text width=3cm
}
}
%---------設置最大節點䥗度------------%

\begin{document}
\maketitle
\thispagestyle{empty}
\clearpage
\tableofcontents
\setcounter{page}{0}
\clearpage

\pagestyle{fancy}
\rhead{DP}

\section{序言}
DP是藉由紀錄小問題和把大問題拆成需多的小問題來節省時間的一個演算法,
作用十分廣泛,
基本上Greedy可以解的題目DP都可以解,
但DP可以解的題目Greedy不一定可以解,
且DP可以把一些窮舉需要$O(2^N)$的問題縮短到$O(N^K)$其中$K\in \mbox{常數}$
\section{費氏數列}
費氏數列想必大家都曾經聽聞了
$$
\begin{cases}
    f_0 = 1 \\
    f_1 = 1 \\
    f_n = f_{n-1} + f{n-2} \qquad n\geq 2
\end{cases}
$$
接下來會介紹幾種解費氏數列的方式。

\subsection{遞迴}

\begin{lstlisting}[language=C++,escapeinside=``]
func f( int n ) {
    if( n == 0 or n == 1)
        return 1
    else 
        return f(n-1)+f(n-2);

}
\end{lstlisting}
\def\GlobalxShift{1cm}
\def\GlobalyShift{1cm}

\begin{figure}[H]
\begin{tikzpicture}
\node (f_4) [node, color=blue] {$f_4$};
\node (f_4_3) [node,below of=f_4, yshift={-\GlobalyShift}, xshift={2*\GlobalxShift}, color=blue!50] {$f_3$};
\node (f_4_2) [node,below of=f_4, yshift={-\GlobalyShift}, xshift={-2*\GlobalxShift}, color=red!50] {$f_2$};
\node (f_4_3_2) [node, below of=f_4_3, yshift={-\GlobalyShift}, xshift={\GlobalxShift}, color=red!50] {$f_2$};
\node (f_4_3_1) [node, below of=f_4_3, yshift={-\GlobalyShift}, xshift={-\GlobalxShift}, color=green] {$f_1$};
\node (f_4_2_1) [node, below of=f_4_2, yshift={-\GlobalyShift}, xshift={\GlobalxShift}, color=green] {$f_1$};
\node (f_4_2_0) [node, below of=f_4_2, yshift={-\GlobalyShift}, xshift={-\GlobalxShift}, color=red] {$f_0$};

\draw [arrow] (f_4) -- (f_4_3);
\draw [arrow] (f_4) -- (f_4_2);
\draw [arrow] (f_4_3) -- (f_4_3_1);
\draw [arrow] (f_4_3) -- (f_4_3_2);
\draw [arrow] (f_4_2) -- (f_4_2_0);
\draw [arrow] (f_4_2) -- (f_4_2_1);
\end{tikzpicture}
\end{figure}

\noindent 可以發現,我們重複計算了很多次一樣的值,
複雜度直接爆炸。\\
時間複雜度$O(\phi^N)$

\subsection{bottom-down}
其實只要多開個陣列來記錄就可以達到很快的效能了
\begin{lstlisting}[language=C++ ,escapeinside=``]
int dp[N]
dp[0] <- 1
dp[1] <- 1

func f( int N ) {
    if( dp[N] != 0 ) 
        return dp[ N ];
    else 
        dp[ N ] = f(N-1) + f(N-2);
        return dp[ N ];
}
\end{lstlisting}
可以發現,
這樣子每一個$f_i$只會被計算到一次,
然後最多只會有$N$個格子,
所以複雜度會是$O(N)$
\subsection{bottom-up}
雖然遞迴的複雜度是$O(N)$,
但他的常數實在有點大,
所以我們為什麼不嘗試看看直接用$for$寫呢?
根據遞迴公式,我們便可以直接使用$for$迴圈來寫這一題。
$$
\begin{cases}
    f_0 = 1 \\
    f_1 = 1 \\
    f_n = f_{n-1} + f{n-2} \qquad n\geq 2
\end{cases}
$$

\begin{lstlisting}[language=C++, escapeinside=``]
int dp[ N ] 
dp[ 0 ] <- 1
dp[ 1 ] <- 1

for i in [2, N] 
    dp[ i ] = dp[ i-1 ] + dp[ i-2 ] 
\end{lstlisting}

這樣子的時間複雜度也會是$O(N)$不過常數小了許多。
\section{經典題型}
\subsection{LCS}
\subsection{LIS}
\subsection{回文子序列}
\section{背包問題}
\subsection{0/1背包}
\subsection{無限背包}
\subsection{有限背包}
\subsubsection{O(NMK)}
\subsubsection{O(NMlogK)}
\subsubsection{O(NM)}

\end{document}
